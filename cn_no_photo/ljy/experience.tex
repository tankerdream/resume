%----------------------------------------------------------------------------------------
%	SECTION TITLE
%----------------------------------------------------------------------------------------

\cvsection{实践经历}

%----------------------------------------------------------------------------------------
%	SECTION CONTENT
%----------------------------------------------------------------------------------------




\begin{cventries}

%------------------------------------------------

\cventry
{面向IoT的数据处理平台,包括多协议M2M即时通信、数据存储、数据共享以及离线、在线数据处理。} % Job title
{物联网云计算平台} % Organization
{项目负责人} % Location
{2015.10 -- 至今} % Date(s)
{ % Description(s) of tasks/responsibilities
\begin{cvitems}
\item {主导完成项目的需求分析、技术选型和API设计,负责项目管理和团队协作}
\item {采用TDD开发流程,基于Node.js、Redis和MongoDB,处理CoAP、MQTT、Socket.IO和HTTP RESTful请求,实现多协议M2M和IoT数据即时处理分析}
\item {开发Python、Node.js、Web、Matlab等环境下的客户端,简化硬件入网流程,实现设备通信、权限控制、资源发现、数据上传等功能}
\item{采用Docker和Kubernetes部署项目到本地和阿里云ECS}
\end{cvitems}
}

\cventry
{研究移动电商真实的行为数据,构建商品推荐模型,为移动用户精准推荐合适的内容。} % Job title
{阿里巴巴移动推荐算法大赛} % Organization
{算法实现} % Location
{2015.5 -- 2015.6} % Date(s)
{ % Description(s) of tasks/responsibilities
\begin{cvitems}
\item {参与算法设计,基于ensemble框架,将GBDT算法、RF算法和LR算法融合在一起,有效地提高了整个模型的泛化能力和鲁棒性}
\item {基于MapReduce和MySQL实现算法流程和测试流程,包括数据预处理、特征提取、特征筛选、模型选择、参数调整、模型融合等}
\end{cvitems}
}

%------------------------------------------------

\cventry
{实现无线传感网络的自供能和声源识别、声源定位。} % Job title
{分布式低功耗自供能无线传感网络声源定位系统} % Organization
{项目负责人} % Location
{2015.06 -- 2015.10} % Date(s)
{ % Description(s) of tasks/responsibilities
\begin{cvitems}
\item {基于6LowPAN完成传感器节点自组网,通过CoAP和网关转换实现传感器网络与互联网的对接}
\item {通过SolidWorks和3D打印完成节点的结构设计和样机制作,实现太阳能、风能、温差综合利用和自供能节点连续工作}
\item{设计实现Web定位界面,采用过零间隔点算法对声源目标进行识别,通过基于能量的声源定位算法和百度地图API实现目标的实时定位}
\end{cvitems} 
}

\cventry
{搜集并存储人体泛在能源,实现系统指标、生理指标的实时监控。} % Job title
{穿戴式俘能及生理指标监控系统} % Organization
{项目负责人} % Location
{2015.04 -- 2016.05} % Date(s)
{ % Description(s) of tasks/responsibilities
\begin{cvitems}
\item {设计并实现基于太阳能、温差、机械能穿戴式俘能系统,使用Raspberry Pi 和Contiki采集并传输人体心率、血氧浓度、体温以及设备能耗指标}
\item {开发Web指标监控界面,多图表实时显示人体生理指标}
\end{cvitems} 
}

\end{cventries}


